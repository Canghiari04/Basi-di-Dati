\documentclass{article}

\usepackage[utf8]{inputenc}
\usepackage[T1]{fontenc}
\usepackage{lipsum}
\usepackage{graphicx}
\usepackage{amsmath}
\usepackage[margin=1in]{geometry}
\usepackage{titlesec}
\usepackage{enumitem}

\titleformat{\section} 
{\LARGE\bfseries}{\thesection}{1em}{}

\titleformat{\subsection} 
{\Large\bfseries}{\thesection}{1em}{}

\begin{document}

\pagestyle{empty}

\section*{Raccolta dei requisiti} 
\large

La \textbf{raccolta/analisi} dei requisiti consiste nella completa \textbf{individuazione dei problemi} che il sistema informativo da realizzare deve risolvere e \textbf{le caratteristiche che il sistema software deve avere}.\\
Le due caratteristiche fondamentali sono quelle \textbf{dei dati}, ovvero le \textbf{informazioni ed i vari vincoli}, e quelle delle \textbf{applicazioni}.\\
Queste informazioni possono essere raccolte da diverse fonti:
\begin{itemize}[label={-}, leftmargin=1cm]
    \itemsep0em
    \item \textbf{Utenti dell'applicazione}: tramite interviste con i committenti e documentazione scritta
    \item \textbf{Documentazione esistente}: normative esistenti, procedure aziendali e regolamenti interni
    \item \textbf{Realizzazioni/applicazioni preesistenti}\\
\end{itemize}
Si ipotizzi di voler realizzare un sistema informativo per una società che eroga corsi di formazione. Bisogna capire \textbf{quali dati devono essere gestiti} e quali \textbf{operazioni sui dati devono essere consentite}.\\
Un primo approccio è quello di produrre un \textbf{documento di specifica}. Essendo il linguaggio naturale spesso fonte di ambiguità e fraintendimenti, una \textbf{buona prassi per la redazione} di un documento di specifica può essere:
\begin{itemize}[label={-}, leftmargin=1cm]
    \itemsep0em
    \item Scegliere il corretto \textbf{livello di astrazione}
    \item Standardizzare la \textbf{struttura delle frasi}
    \item Evitare \textbf{frasi contorte}
    \item Individuare \textbf{omonimi/sinonimi}
    \item Esplicitare il \textbf{riferimento tra i termini}
\end{itemize}
Potrebbe anche essere utile \textbf{decomporre} il testo di specifica in \textbf{frasi omogenee}, relative agli stessi concetti.\vspace*{14pt}\\
Dopo aver prodotto un \textit{documento di specifica}, si procede con la costruzione di un \textbf{glossario dei termini}, contenete descrizione, sinonimi e collegamenti.\vspace*{14pt}\\
Per concludere, si definiscono le \textbf{operazioni sui dati}. Questo step è utile per:
\begin{itemize}[label={-}, leftmargin=1cm]
    \itemsep0em
    \item Verificare la \textbf{completezza dei modelli} sviluppati in fase di progettazione logica e concettuale
    \item Valutare le \textbf{prestazioni dei modelli} sviluppati in fase di progettazione logica e concettuale
    \item Fornire \textbf{linee guida per l'implementazione} dei dati, come ad esempio l'utilizzo di stored procedures per le operazioni
\end{itemize}
\end{document}