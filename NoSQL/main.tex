\documentclass{article}

\usepackage[utf8]{inputenc}
\usepackage[T1]{fontenc}
\usepackage{lipsum}
\usepackage{graphicx}
\usepackage{amsmath}
\usepackage[margin=1in]{geometry}
\usepackage{titlesec}
\usepackage{enumitem}

\titleformat{\section} 
{\LARGE\bfseries}{\thesection}{1em}{}

\titleformat{\subsection} 
{\Large\bfseries}{\thesection}{1em}{}

\begin{document}

\pagestyle{empty}

\section*{Introduzione ai DBMS NoSQL} 
\large

Il movimento \textbf{NoSQL} promuove l'adozione di DBMS \textbf{non basati sul modello relazionale}. Il termine NoSQL viene usato per lo più nell'accezione \textbf{NoT Only SQL}.\\
Alcune proprietà dei sistemi NoSQL sono:
\begin{itemize}[label={-}, leftmargin=1cm]
    \item Database \textbf{distribuiti}
    \item Tool \textbf{open-source}
    \item Non dispongono di \textbf{schema}
    \item Non supportano operazioni di \textbf{join}
    \item Non implementano le proprietà \textbf{ACID} delle transazioni
    \item Sono \textbf{scalabili} orizzontalmente
    \item Sono in grado di gestire \textbf{grandi moli di dati}
    \item Supportano le \textbf{repliche} dei dati
\end{itemize}
Si osservino ora le motivazioni principali legate alla diffusione dei databse \textit{NoSQL}:\vspace{14pt}\\
\textit{Gestione dei Big-Data}\\
I \textbf{Big-Data} sono grandi moli di dati, eterogenei, destrutturati, difficili da gestire attraverso tecnologie tradizionali, come i \textit{RDBMS}. Il termine \textit{big-data} è oggi usato sia per denotare tipologie di dati, sia le \textbf{tecnologie} e i \textbf{tool} di gestione degli stessi.\\
Nel contesto dei big-data, osserviamo dei concetti fondamentali tramite le quattro \textbf{V}:
\begin{itemize}[label={-}, leftmargin=1cm]
    \item \textbf{Volume}: grossa mole di dati. Alcuni esempi possono essere i dati di esperimenti scientifici, sensoristica, IoT
    \item \textbf{Velocità}: \textit{stream} continuo di dati. Un esempio potrebbero essere i sistemi healt-care
    \item \textbf{Varietà}: dati eterogenei, multi sorgente. Un esempio potrebbero essere i social media
    \item \textbf{Valore}: possibilità di estrarre conoscenza dai dati. Alcuni esempi possono essere il \textbf{data mining}, ovvero delle tecniche di \textbf{apprendimento computerizzato} per analizzare ed estrarre \textbf{conoscenze} da collezioni di dati\\
\end{itemize}
\textit{Limitazioni del modello relazionale}\\
Sono presenti tre limitazioni principali:
\begin{itemize}[label={-}, leftmargin=1cm]
    \item Il modello relazionale presuppone implicitamente la presenza di \textbf{dati strutturati} in rappresentazione tabellare. Cosa accade se i dati non si presentano in tale forma?
    \item Alcune operazioni non possono essere implementate in SQL, come la memorizzazione di un grafo e il calcolo del percorso minimo tra due punti
    \item La scalabilità \textit{orizzontale} dei DBMS relazioni. La \textbf{scalabilità} è la capacità di un sistema di migliorare le proprie prestazioni per un certo carico di lavoro, quando vengono aggiunte nuove risorse al sistema. Per quanto riguarda la scalabilità \textit{verticale}, si parla di aggiungere più potenza di calcolo ad i nodi che gestiscono il databse. Invece, per scalabilità \textit{orizzontale} si intende aggiungere più nodi al cluster
\end{itemize}
\textit{Teorema CAP}\\
Il teorema di \textbf{Brewer} (\textbf{CAP Theorem}) afferma che \textbf{un sistema distribuito può soddisfare \underline{al massimo solo due} delle tre proprietà elencate successivamente}:
\begin{itemize}[label={-}, leftmargin=1cm]
    \item \textbf{Consistency}: tutti i nodi della rete vedono gli stessi dati
    \item \textbf{Availability}: il servizio è sempre disponibile
    \item \textbf{Partion Tolerance}: il servizio continua a funzionare correttamente anche in presenza di perdita di messaggi o di partizionamenti della rete
\end{itemize}
Si osservi ora i diversi casi, combinando le tre proprietà in coppie:
\begin{itemize}[label={-}, leftmargin=1cm]
    \item \textbf{Consistency + Availability (CA)}: si parla di \textbf{No Partition Tolerance}, dove il sistema \textit{non funziona correttamente} in caso di perdita di messaggi
    \item \textbf{Availability + Partion Tolerance (AP)}: si parla di \textbf{No Consistency}, dove si trovano repliche del dato non aggiornate
    \item \textbf{Consistency + Partion Tolerance (CP)}: si parla di \textbf{No Availability}, dove la query non produce risposta\\
\end{itemize}
Ritornando al concetto di DBMS NoSQL, esistono alcune proprietà base:
\begin{itemize}[label={-}, leftmargin=1cm]
    \item \textbf{Basically Available}: i nodi del sistema distribuito possono essere soggetti a guasti, ma il servizio è sempre disponibile
    \item \textbf{Soft State}: la consistenza dei dati non è garantita in ogni istante
    \item \textbf{Eventually Consistent}: il sistema diventa consistente dopo un certo intervallo di tempo, se le attività di modifica dei dati cessano\\
\end{itemize}
Il termine NoSQL identifica \textbf{una varietà di DBMS non relazionali}, basati su \textbf{modelli logici differenti}:\vspace{14pt}\\
\textit{Databse \textbf{chiave-valore}}\\
I dati di un databse sono come liste di coppie \textit{chiave/valore}, come per array associativi o dizionari. La \textbf{chiave} è un valore univoco per operazioni di ricerca, mentre il \textbf{valore} è il valore associato alla chiave. Alcuni esempi possono essere \textbf{Project Voldemort} e \textbf{BerkeleyDB}.\vspace{14pt}\\
\textit{Databse \textbf{document-oriented}}\\
Avviene una gestione di dati \textbf{eterogeneei} e \textbf{complessi}. Sono scalabili \textbf{orizzontalmente}, con supporto per partizionamento dei dati in sistemi distribuiti. I \textbf{documenti} sfruttano il concetto di coppie chiave/valore, tramite file \textit{JSON}. Questa gestione dei databse fornisce funzionalità per aggregazione/analisi dei dati. Alcuni esempi possono essere \textbf{MongoDB} e \textbf{CouchDB}.\vspace{14pt}\\
\textit{Databse \textbf{column-oriented}}\\
I dati vengono organizzati su colonne anziché delle righe. Vengono utilizzate le column family. Le \textbf{column family} sono contenitori di colonne, dove ogni column family è scritta su un file diverso. Ogni riga dispone di una chiave primaria, detta \textbf{row key}. E' uno schema \textbf{flessibile}, con maggiore efficienza nello storage e maggiore possibilità di \textbf{comprensione dei dati}. Alcuni esempi possono essere \textbf{HBase} e \textbf{Cassandra}.\vspace{14pt}\\
\textit{Databse \textbf{graph-oriented}}\\
I dati vengono strutturati sotto forma di grafi, dove ogni nodo ed arco ha diversi attributi. Alcuni esempi possono essere \textbf{Neo4J} e \textbf{Titan}.
\end{document}